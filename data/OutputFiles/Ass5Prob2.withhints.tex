% Created 2013-07-24 Wed 11:22
\documentclass[11pt]{article}
\usepackage[utf8]{inputenc}
\usepackage[T1]{fontenc}
\usepackage{fixltx2e}
\usepackage{graphicx}
\usepackage{longtable}
\usepackage{float}
\usepackage{wrapfig}
\usepackage{soul}
\usepackage{textcomp}
\usepackage{marvosym}
\usepackage{wasysym}
\usepackage{latexsym}
\usepackage{amssymb}
\usepackage{hyperref}
\tolerance=1000
\providecommand{\alert}[1]{\textbf{#1}}

\title{Ass5Prob2.withhints}
\author{Yoav Freund}
\date{\today}
\hypersetup{
  pdfkeywords={},
  pdfsubject={},
  pdfcreator={Emacs Org-mode version 7.9.3f}}

\begin{document}

\maketitle

\setcounter{tocdepth}{3}
\tableofcontents
\vspace*{1cm}
\section{Part   1}
\label{sec-1}


\#\# Probability of having an empty bin \#\#

Suppose there are [`m`] balls and [`n`] bins. What is the chance that there is an empty bin if [`m = n`] ?

Intuitively, one would expect this to be pretty close to 1 (that is, pretty much certain), because the complementary event -- no empty bins -- would occur only if every single bin received exactly one ball. Let's analyze this latter probability. The event whose probability we want to upper bound is the event that no two balls land in the same bin.

[``A=\left\{ (b$_1$,b$_2$,\ldots,b$_n$) | $\forall$ 1\leq i<j \leq n, b$_i$ $\neq$ b$_j$ \right\}``]

The probability that the first ball does not cause any bin to have more than one ball is \textbf{[  1]\{1\}}
\section{Answers of student jmc001. 10 attempts lasting 15.4 minutes, ending in sucess}
\label{sec-2}
\subsection{All but last attempt}
\label{sec-2-1}

  0:      1      0.0    0       1-(1/n)
  1:      1      0.1    0       n-1
  2:      1      0.2    1       1
  3:      1      0.3    0       n
  4:      1      1.1    0       0
  5:      1      2.9    0       100
  6:      1     11.2    0       n-1
  7:      1     11.3    1       1
  8:      1     15.3    0       n-1
\subsection{9:   1     15.4    1       1}
\label{sec-2-2}
\section{Part   2}
\label{sec-3}

.

Without loss of generality, let us assume the first ball is placed in
the first bin. Then we place the second ball. In order to ensure every
bin has only one ball, we cannot place the second ball in the first
bin because it already has the first ball. This leaves us with the
other [`n-1`] bins to put the second ball in. The probability that the
second ball avoids the first bin is \textbf{[ 2]\{(1-1/n)\}}
\section{Answers of student jmc001.  5 attempts lasting 13.8 minutes, ending in sucess}
\label{sec-4}
\subsection{All but last attempt}
\label{sec-4-1}

  0:      2      9.0    0       m-1
  1:      2      9.9    0       n-1/m
  2:      2     10.5    0       1-(2/n)
  3:      2     22.1    0       n-1
\subsection{4:   2     22.9    1       1-(1/n)}
\label{sec-4-2}
\section{Part   3}
\label{sec-5}

.

Again, without loss of generality, we assume the second ball is placed
in the second bin. For the third ball, it can not be placed in the
first or the second bin. The probability that the third ball avoids
these two bins is

\textbf{(A)} \textbf{[  3]\{(1-2/n)\}}
\section{Answers of student jmc001.  3 attempts lasting 0.2 minutes, ending in sucess}
\label{sec-6}
\subsection{All but last attempt}
\label{sec-6-1}

  0:      3     27.9    0       1-(1/n)
  1:      3     28.1    0       1-(3/n)
\subsection{2:   3     28.1    1       1-(2/n)}
\label{sec-6-2}
\section{Part   4}
\label{sec-7}

.

More generally, the probability that the placement of the [`k`] th
ball does not cause any bin to have more than one ball is

\textbf{(B)} \textbf{[  4]\{(1-(k-1)/n)\}}
\section{Answers of student jmc001. 15 attempts lasting 31.0 minutes, ending in sucess}
\label{sec-8}
\subsection{All but last attempt}
\label{sec-8-1}

  0:      4     34.8    0       1-(n-1/n)
  1:      4     34.9    0       1-((n-1)/n)
  2:      4     38.0    0       n!
  3:      4     38.7    0       1/n
  4:      4     44.1    0       1-n!
  5:      4     44.5    0       n!/(k!(n-k!))
  6:      4     47.9    0       (n-1)/n
  7:      4     49.1    0       1-n!/n$^n$
  8:      4     49.2    0       1-n!/n
  9:      4     51.5    0       1-(k/n)
 10:      4     51.9    0       1/(n-k)
 11:      4     60.8    0       k/n!
 12:      4     61.2    0       (k-1)/n
 13:      4     65.5    0       1-(k-1/n)
\subsection{14:   4     65.7    1       1-((k-1)/n)}
\label{sec-8-2}
\section{Part   5}
\label{sec-9}

.

Together, after placing all [`m=n`] balls, the probability that each
bin has exactly one ball is

\textbf{(C)} \textbf{[  5]\{n!/n$^n$\}}
\section{Answers of student jmc001.  2 attempts lasting 26.9 minutes, ending in sucess}
\label{sec-10}
\subsection{All but last attempt}
\label{sec-10-1}

  0:      5     11.2    0       1/2^(n/2)
\subsection{1:   5     38.0    1       n!/n$^n$}
\label{sec-10-2}
\section{Part   6}
\label{sec-11}

 (write the answer in [`n`] only).

This probability is miniscule. To show this, we need to upper bound \textbf{(C)}.

We start by upper bounding \textbf{(B)}.  Recall the inequality [`1+x \leq e$^x$`]. Plugging [`-\frac{k-1}{n}`] in for [`x`], gives us an upper bound on \textbf{(B)}
that is 

\textbf{(D)} \textbf{[  6]\{Formula(``e^(-(k-1)/n)'')\}}
\section{Answers of student jmc001.  3 attempts lasting 2.7 minutes, ending in sucess}
\label{sec-12}
\subsection{All but last attempt}
\label{sec-12-1}

  0:      6     67.7    0       e^((k-1)/n)
  1:      6     70.4    0       1-((k-1)/n)
\subsection{2:   6     70.4    1       e^(-(k-1)/n)}
\label{sec-12-2}
\section{HINT:}
\label{sec-13}

In part \textbf{(D)} we upper bounded the probability  that the [`k`]th ball
lands in an empty bin. We now want to upper bound the probability
that all [`n`] balls land in an empty bin. Note that the probability
that the [`k`]th call falls in an empty bin depends on the fact that
it is the \textbf{[`k`]th} ball, but \textbf{not} on the locations of the [`k-1`]
occupied bins. Let's denote by [`P$_k$= 
\section{HINT:}
\label{sec-14}

The product notation is similar to the sum notation, but with
multiplication taking the place of addition. In other words, just like
[`$\sum$$_{\mathrm{i=1}}$$^3$ i$^2$ = 1$^2$+2$^2$+3$^2$ = 1+4+9 = 14`]
we have that
\[$\prod$$_{\mathrm{i=1}}$$^3$ i$^2$ = 1$^2$ \texttimes{} 2$^2$ \texttimes{} 3$^2$ = 1 \texttimes{} 4 \texttimes{} 9
= 36`]

$\sum_{i=1}^\infty \frac{1}{i^2}$

Write the expanded expression for [`$\prod$$_{\mathrm{i=3}}$$^4$ (1-i/5)`]
[_________]\{''(1-3/5)(1-4/5)''\}
\section{HINT:}
\label{sec-15}

The product of two exponentials with the same basis is equal to the
exponent of the sum. For example:
[`2$^5$ \texttimes{} 2$^3$ = 2$^{\mathrm{5+3}}$ = 256`]
[`e$^{\pi}$ \texttimes{} e$^2$ = e$^{\mathrm{2\pi}}$]

Write the following expression in the form [`e^($\sum$ $\cdot$)`]

[` $\prod$$_{\mathrm{k=2}}$$^n$ e^(-(k-1)/n)'] = [_______________]\{''e^($\sum$$_{\mathrm{k=2}}$$^n$
(k-1)/n)''\}
\section{Part   7}
\label{sec-16}

.

We now want to derive an upper bound on \textbf{(C)} which is the probability
that each of the [`n`] balls is placed in a different bin. To do this
we take the product [`$\prod$$_{\mathrm{k=1}}$$^n$ $\exp$(-(k-1)/n)`] simplifying this
expression gives

[`P(A) \leq `] \textbf{[  7]\{Formula(``e^(-(n-1)/2)'')\}}
\section{Answers of student jmc001. 30 attempts lasting 54.0 minutes, ending in failure}
\label{sec-17}
\subsection{All but last attempt}
\label{sec-17-1}

  0:      7     67.7    0       n!/n$^n$
  1:      7     70.4    0       e^(-(k/n))
  2:      7     70.7    0       e^(-(k!/n$^n$))
  3:      7     70.8    0       e^(k!/n$^n$)
  4:      7     72.5    0       e^(k$^n$)
  5:      7     73.7    0       e^(n!/n$^n$)
  6:      7     76.2    0       e^(n!/k!(n-k)!)
  7:      7     76.4    0       e^(n!/(k!(n-k!))
  8:      7     76.5    0       e^(n!/(k!(n-k!)))
  9:      7     81.4    0       1/k^(k/2)
 10:      7     81.7    0       k^(k/2)
 11:      7     82.4    0       1/ek$^k$
 12:      7     82.7    0       1/(e(k$^k$))
 13:      7     83.6    0       (1-(1/n))$^n$-k
 14:      7     83.8    0       (1-(1/n))^(n-k)
 15:      7     84.4    0       1/n$^k$
 16:      7     84.5    0       1/n$^2$
 17:      7     87.1    0       [e^(-(k-1)/n)]*(n!/n$^n$)
 18:      7     87.6    0       [e^(-(k-1)/n)]*[1-((k-1)/n)]
 19:      7     88.8    0       [e^(-(k-1)/n)]*e$^k$
 20:      7     94.1    0       (-(k-1)/n)*[n!/n$^n$]
 21:      7     94.4    0       e$^{\mathrm{(-(k-1)/n)*[n!/n^n]}}$
 22:      7     99.1    0       [e^(n!/n$^n$)]*[e^(-(k-1)/n)]
 23:      7     110.2   0       [(ne/k)$^k$]*(1/n)$^k$
 24:      7     111.3   0       [(e/k)$^k$]*1/(1-(e/k))
 25:      7     111.6   0       [(e/k)$^k$]*[1/(1-(e/k))]
 26:      7     113.7   0       e^(-[m(m-1)]/2n)
 27:      7     121.3   0       [1-((k-1)/n)]*[n!/n$^n$]
 28:      7     121.5   0       e^[1-((k-1)/n)]*[n!/n$^n$]
\subsection{29:   7     121.7   0       e$^{\mathrm{[1-((k-1)/n)]*[n!/n^n]}}$}
\label{sec-17-2}

\end{document}
